%%%%%%%%%%%%%%%%%%%%%%%%%%%%%%%%%%%%%%%%%
% Structured General Purpose Assignment
% LaTeX Template
%
% This template has been downloaded from:
% http://www.latextemplates.com
%
% Original author:
% Ted Pavlic (http://www.tedpavlic.com)
%
% Note:
% The \lipsum[#] commands throughout this template generate dummy text
% to fill the template out. These commands should all be removed when 
% writing assignment content.mus
%
%%%%%%%%%%%%%%%%%%%%%%%%%%%%%%%%%%%%%%%%%

%----------------------------------------------------------------------------------------
%	PACKAGES AND OTHER DOCUMENT CONFIGURATIONS
%----------------------------------------------------------------------------------------

\documentclass{article}

%\usepackage[brazilian]{babel}
\usepackage[utf8]{inputenc}
\usepackage{fancyhdr} % Required for custom headers
\usepackage{lastpage} % Required to determine the last page for the footer
\usepackage{extramarks} % Required for headers and footers
\usepackage{graphicx} % Required to insert images
\usepackage{float}
\usepackage{listings}
\usepackage{amsmath}
\usepackage[colorlinks=true, pdfborder={0 0 0}, urlcolor=blue, linkcolor=black]{hyperref}

\graphicspath{ {img/} }

% Margins
\topmargin=-0.45in
\evensidemargin=0in
\oddsidemargin=0in
\textwidth=6.5in
\textheight=9.0in
\headsep=0.25in 

\linespread{1.1} % Line spacing

% Set up the header and footer
\pagestyle{fancy}
\lhead{\hmwkAuthorName} % Top left header
%\chead{\hmwkClass\ (\hmwkClassInstructor\ \hmwkClassTime): \hmwkTitle} % Top center header
\rhead{\hmwkClass: \hmwkTitle} % Top center header
%\rhead{\firstxmark} % Top right header
\lfoot{\lastxmark} % Bottom left footer
\cfoot{} % Bottom center footer
\rfoot{Página\ \thepage\ de\ \pageref{LastPage}} % Bottom right footer
\renewcommand\headrulewidth{0.4pt} % Size of the header rule
\renewcommand\footrulewidth{0.4pt} % Size of the footer rule

\setlength\parindent{0pt} % Removes all indentation from paragraphs

%----------------------------------------------------------------------------------------
%	DOCUMENT STRUCTURE COMMANDS
%	Skip this unless you know what you're doing
%----------------------------------------------------------------------------------------

% Header and footer for when a page split occurs within a problem environment
\newcommand{\enterProblemHeader}[1]{
\nobreak\extramarks{#1}{#1 continues in next page\ldots}\nobreak
\nobreak\extramarks{#1 (continuation)}{#1 continues in next page\ldots}\nobreak
}

% Header and footer for when a page split occurs between problem environments
\newcommand{\exitProblemHeader}[1]{
\nobreak\extramarks{#1 (continuation)}{#1 continues in next page\ldots}\nobreak
\nobreak\extramarks{#1}{}\nobreak
}

\setcounter{secnumdepth}{0} % Removes default section numbers
\newcounter{homeworkProblemCounter} % Creates a counter to keep track of the number of problems

\newcommand{\homeworkProblemName}{}
\newenvironment{homeworkProblem}[1][\arabic{homeworkProblemCounter}]{ % Makes a new environment called homeworkProblem which takes 1 argument (custom name) but the default is "Problem #"
\stepcounter{homeworkProblemCounter} % Increase counter for number of problems
\renewcommand{\homeworkProblemName}{#1} % Assign \homeworkProblemName the name of the problem
\section{\homeworkProblemName} % Make a section in the document with the custom problem count
\enterProblemHeader{\homeworkProblemName} % Header and footer within the environment
}{
\exitProblemHeader{\homeworkProblemName} % Header and footer after the environment
}

\newcommand{\problemAnswer}[1]{ % Defines the problem answer command with the content as the only argument
\noindent\framebox[\columnwidth][c]{\begin{minipage}{0.98\columnwidth}#1\end{minipage}} % Makes the box around the problem answer and puts the content inside
}

\newcommand{\homeworkSectionName}{}
\newenvironment{homeworkSection}[1]{ % New environment for sections within homework problems, takes 1 argument - the name of the section
\renewcommand{\homeworkSectionName}{#1} % Assign \homeworkSectionName to the name of the section from the environment argumen
\subsection{\homeworkSectionName} % Make a subsection with the custom name of the subsection
\enterProblemHeader{\homeworkProblemName\ [\homeworkSectionName]} % Header and footer within the environment
}{
\enterProblemHeader{\homeworkProblemName} % Header and footer after the environment
}
   
%----------------------------------------------------------------------------------------
%	NAME AND CLASS SECTION
%----------------------------------------------------------------------------------------

\newcommand{\hmwkTitle}{Proposal for Bachelor Thesis 02} % Assignment title
\newcommand{\hmwkDueDate}{26 October 2014} % Due date
\newcommand{\hmwkClass}{} % Course/class
\newcommand{\hmwkClassFull}{} % Course/class
\newcommand{\hmwkClassTime}{} % Class/lecture time
\newcommand{\hmwkClassInstructor}{} % Teacher/lecturer
\newcommand{\hmwkAuthorName}{William Bombardelli da Silva} % Your name

%----------------------------------------------------------------------------------------
%	TITLE PAGE
%----------------------------------------------------------------------------------------

\title{
%\vspace{2in}
\Large\textmd{\textbf{\hmwkClassFull}}\\
\normalsize{\textbf{\hmwkTitle}}\\
\normalsize\vspace{0.1in}\small{\hmwkDueDate}\\
%\vspace{0.1in}
\large{\textit{\hmwkClassInstructor\ \hmwkClassTime}}
%\vspace{3in}
}

\author{\textbf{\hmwkAuthorName}}
\date{} % Insert date here if you want it to appear below your name

%----------------------------------------------------------------------------------------

\begin{document}

%\maketitle
{\centering
\Large\textmd{\textbf{\hmwkClassFull}}\\
\normalsize{\textbf{\hmwkTitle}}\\
\normalsize\vspace{0.1in}\small{\hmwkDueDate}\\
%\vspace{0.1in}
\large\textbf{\hmwkAuthorName}
%\large{\textit{\hmwkClassInstructor\ \hmwkClassTime}}\\

}
%----------------------------------------------------------------------------------------
%	TABLE OF CONTENTS
%----------------------------------------------------------------------------------------

%\setcounter{tocdepth}{1} % Uncomment this line if you don't want subsections listed in the ToC

%\newpage
%\tableofcontents
%\newpage

% To have just one problem per page, simply put a \clearpage after each problem

\begin{homeworkProblem}[Synchronizing a Network of Models with ATL]
	\begin{section}{Introduction}
	This document intents to describe the proposal number 02 for bachelor thesis by William Bombardelli da Silva, student of Informatik Bachelor in the Technische Universität Berlin, student number 364927.
	
	The goal of this bachelor thesis is to report the current state of research and to develop new contributions both theoretical and practical. By being so, the thesis aims to investigate a problem of software model synchronization in the context of Model-driven Engineering. More specifically the problem of building a network of software meta-models maintaining the relations (or transformations) between them correct, this means maintaining the whole network synchronized.
	\end{section}
	
	\begin{section}{Key-words}
	Model Synchronization, Iterative Model Transformation, Model Transformation, Model-driven Engineering, Software Engineering.
	\end{section}	
	
	\begin{section}{Theme Description}
	Recent techniques of software engineering have been using the concept of software models in the construction of software systems. According to [Czarnecki and Helsen 2006] "\textit{Models are system abstractions that allow developers and other stakeholders to effectively address concerns, such as answering a question about the system or effecting a change}”. By defining a model as a system abstraction, it becomes clear, that a software system might have several models abstracted from it, each one representing certain aspects of the whole system. These models also have relations between themselves, in the sense that they all are supposed to describe the actual system consistently by not presenting logical contradictions. Here examples of models are \emph{UML class diagram}, \emph{Use Cases}, or even the source-code itself.
	
	The possible diversity of models in a system and the vast number of their relationships suggests the creation of a network of models for a system. Where each component of the network is a different model, and two components are interconnected if, and only if their respective models have a relationship between them. One can see this network alternatively as a graph, whereas each vertex \emph{v} represents a model, and an edge connecting $v_i$ and $v_j$ exists if, and only if there is a relationship defined between both models \emph{i} and \emph{j}.
	
	The goal of this bachelor thesis is then to create an example network meta-models (M2-level models, see [QUOTE]) defining the meta-models and their relations using the transformation language \emph{ATL}. After having this network ready, the synchronization algorithm used to maintain all the relations in the network correct (i.e. keeping the whole network synchronized) is supposed to be developed. As a basis for such algorithm the algorithm proposed in [Xiong 2007] can be used.
	
	In the end of the development of this thesis it is expected a stable version of the synchronization method for a network of models; the arising of inferences of theoretical properties over the network; and the creation of a report comprising the difficulties and the next challenges for the problem. The meta-models used might be narrowed to MOF compliant (see [QUOTE]) meta-models though, including \emph{Java code}, \emph{UML diagrams}, \emph{formal specifications}, among others.
	%TODO: QUOTE MOF
	\end{section}
	
	\begin{section}{Motivation}
	The main motivation for this thesis is the lack of a consistent and clear method for synchronization of a network of models in current literature. In fact the strategy of seeing a software system as a network of different models interconnected by their relations is relatively recent and seems to be very prominent. What creates the motivation for further development of techniques and methods for this new realm.
	
	It is worth to note also, that the contribution of this thesis might help enhancing the quality of current software construction and therefore lessening the number of software problems and errors, what is an endemic problem nowadays. Once that, the use of models in software engineering seems to reduce the occurrence of errors (QUOTE???) of software and that, the broad use of models in industry is still restricted partly because of the lack of a practical method for synchronization of models in a network, like the one proposed here. 
	%TODO: QUOTE???
	
	It is possible that the application of a network of models able to be kept synchronized in software engineering projects be the key to finally bridge the gap between abstract models and concrete models, specially because of the vast possibility to soundly link different models in different levels of abstraction in such a network.
	\end{section}
	
	\begin{section}{State of Current Research}
	A research road-map for model synchronization found in [France 2007] gives an overview on the realm, and an interesting point of view about the challenges. In [Mens 2006] a taxonomy for model transformation is proposed, what helps to carry more precise analysis. In [Czarnecki and Helsen 2006] a survey was driven and a framework for classification of model transformation approaches was presented.
	
	In [Diskin 2015] a taxonomy for a network of models is presented and in [Diskin 2011] a theoretical algebraic basis is proposed. These both works may be used extensively in our further development. Additionally, one can judge by the date of publication of these works, that the topic of working with networks of models is extremely active and is actually the edge of current academic research, what motivates even more the development of this thesis.
	
	The bottom line for the development of the synchronization algorithm can be the one proposed in [Xiong 2007], which is supposed to be used to synchronized two models. The challenge would be then to further evolve it, so that it solves the problems found in synchronizing not only two models but a whole network. Such problems include occurrence of cycles and how to proceed with the forwarding of modifications throughout the net.
	
	For describing the relations the \emph{ATL Transformation Language} is supposed to be used, references can be found in [QUOTE ATL]
	%TODO: QUOTE ATL
	\end{section}
	
	\begin{section}{Concepts Definition}
	Below is a list of necessary basic concepts, that will be used throughout this document.
	
	\textbf{Model:} The definition for software model used is: “Models are system abstractions that allow developers and other stakeholders to effectively address concerns, such as answering a question about the system or effecting a change” [Czarnecki and Helsen 2006]. Examples of models, according to this definition, are \emph{UML diagrams}, \emph{OCL expressions}, \emph{relational database diagrams}, or even source-code.
	
	\textbf{Modeling Language:} For the scope considered here, modeling language is broadly defined as a language used to create models, that can be textual or graphical and may occur in several paradigms. Among them, functional, declarative or operational paradigms.
	
	\textbf{Model Relation:} Model relation here is defined abstractly as every relationship or constraint possible to happen between one source model and one target model.
	
	\textbf{Model Transformation:} Model transformation can be viewed as common data transformation – very common in computer science – with the specificity of dealing with models [Czarnecki and Helsen 2006]. In these terms, one can see model transformation from several points of view, for example as a sequence of operations/modifications over one model; or as a function, whose inputs relate to initial models and the output reflect the updated models. The borders between model transformation and model relation are sometimes fuzzy and both terms might be used interchangeably.
	
	\textbf{Model Synchronization:} The goal of model synchronization is to maintain all relations between the models of a system consistent/correct as updates are performed over them [Diskin 2011]. Model synchronization may be seen as a procedure, a series of model transformations; as a function, which output – e.g. consistent updated set of models – is determined by the inputs – e.g. consistent set of models plus transformations to be executed; or even as a relation between set of models.
	As model synchronization is a relative new field of study, no fixed definition or approach is consensus between researchers, what brings up a variety of possible strategies and conjectures to solve the problem.
	
	\textbf{Network of Models:} A network of models of a system $S$ is an undirected graph $G = (V,E)$, whereas each vertex $v_i \in V$ represents a unique model $i$ of $S$, and an edge $(v_i, v_j)$ exists if, and only if there is a relation defined between both models $i, j \in S$.
	\end{section}
	
	\begin{section}{Problem Definition}
	...
	\end{section}
	
	\begin{section}{Possible Difficulties}
	...
	\end{section}
	
	\begin{section}{Time Schedule}
	\begin{table}[h]
	\centering
	\begin{tabular}{l | l | l }
		\textbf{Duration} & \textbf{Start and End Dates} & \textbf{Activities} \\ \hline
			2 Weeks &	13/10/2015 to 26/10/2015 &	Initial research; definition of theme; finding of literature\\ \hline
			1 Weeks &	27/10/2015 to 02/11/2015 &	Detailed research; write of proposal; definition of scope\\ \hline
			2 Weeks &	03/11/2015 to 16/11/2015 &	Deepening in the theme; sketch of the development\\ \hline
			3 Weeks &	17/11/2015 to 07/12/2015 &	Analysis; design; pre-development phase; review; start of the writing\\ \hline
			5 Weeks &	08/12/2015 to 11/01/2016 &	Development; testing, validation and verification; review; writing\\ \hline
			2 Weeks &	12/01/2016 to 25/01/2016 &	Validation and verification; review; writing\\ \hline
			3 Weeks &	26/01/2016 to 15/02/2016 &	Finalization of writing; review\\ \hline
			1 Weeks &	16/02/2016 to 22/02/2016 &	Preparation of presentation\\ \hline
			1 Weeks &	23/02/2016 to 29/02/2016 &  Final review\\ \hline
	\end{tabular}
	\caption{Plan for the researching, developing and writing of the bachelor thesis. This schedule is organized in weeks, whereas each week has its respective activities planed.}
	\end{table}
	\end{section}
	
	\begin{section}{Bibliography}
	%TODO: [Czarnecki and Helsen 2006]
		Diskin, Zinovy, et al. "Towards a rational taxonomy for increasingly symmetric model synchronization." \textit{Theory and Practice of Model Transformations}. Springer International Publishing, 2014. 57-73.\\
		\\
		Diskin, Zinovy. "Model synchronization: Mappings, tiles, and categories." \textit{Generative and Transformational Techniques in Software Engineering III}. Springer Berlin Heidelberg, 2011. 92-165.\\
		\\
		Zinovy Diskin, Hamid Gholizadeh, Arif Wider, Krzysztof Czarnecki, "A Three-Dimensional Taxonomy for Bidirectional Model Synchronization", \textit{The Journal of Systems \& Software (2015)}, doi: 10.1016/j.jss.2015.06.003 \\
		\\
		France, Robert, and Bernhard Rumpe. "Model-driven development of complex software: A research roadmap." \textit{2007 Future of Software Engineering}. IEEE Computer Society, 2007.\\
		\\
		%\bibitem{Xiong2007}
		Xiong, Yingfei, et al. "Towards automatic model synchronization from model transformations." \textit{Proceedings of the twenty-second IEEE/ACM international con-ference on Automated software engineering}. ACM, 2007.\\
		\\
		Jouault, Frédéric, et al. "ATL: A model transformation tool." \textit{Science of computer programming} 72.1 (2008): 31-39
	\end{section}
\end{homeworkProblem}

\end{document}

