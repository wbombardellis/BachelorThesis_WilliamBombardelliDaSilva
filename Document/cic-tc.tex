% 
% exemplo genérico de uso da classe iiufrgs.cls
% $Id: iiufrgs.tex,v 1.1.1.1 2005/01/18 23:54:42 avila Exp $
%
% This is an example file and is hereby explicitly put in the
% public domain.
%
\documentclass[tuberlin,cic,tc,openright,english]{iiufrgs}
% Para usar o modelo, deve-se informar o programa e o tipo de documento.
% Programas :
% * cic       -- Graduação em Ciência da Computação
% * ecp       -- Graduação em Ciência da Computação
% * ppgc      -- Programa de Pós Graduação em Computação
% * pgmigro   -- Programa de Pós Graduação em Microeletrônica
% * tuberlin  -- Bachelorarbeit entregue na TU Berlin
%
% Tipos de Documento:
% * tc                -- Trabalhos de Conclusão (apenas cic e ecp)
% * diss ou mestrado  -- Dissertações de Mestrado (ppgc e pgmicro)
% * tese ou doutorado -- Teses de Doutorado (ppgc e pgmicro)
% * ti                -- Trabalho Individual (ppgc e pgmicro)
%
% Outras Opções:
% * english    -- para textos em inglês
% * openright  -- Força início de capítulos em páginas ímpares (padrão da
% biblioteca)
% * oneside    -- Desliga frente-e-verso
% * nominatalocal -- Lê os dados da nominata do arquivo nominatalocal.def


% Use unicode
\usepackage[utf8]{inputenc}   % pacote para acentuação

% Necessário para incluir figuras
\usepackage{graphicx}         % pacote para importar figuras
\graphicspath{ {img/} }

\usepackage{times}            % pacote para usar fonte Adobe Times
% \usepackage{palatino}
% \usepackage{mathptmx}       % p/ usar fonte Adobe Times nas fórmulas

\usepackage[alf,abnt-emphasize=bf]{abntex2cite}	% pacote para usar citações abnt

%
% Informações gerais
%
\title{aaa}

\author{Bombardelli da Silva}{William}
% alguns documentos podem ter varios autores:
% \author{Flaumann}{Frida Gutenberg}
% \author{Flaumann}{Klaus Gutenberg}

% orientador e co-orientador são opcionais (não diga isso pra eles :))
\advisor[Prof.~Dr.]{}{}
%\coadvisor[Prof.~Dr.]{Knuth}{Donald Ervin}

% a data deve ser a da defesa; se nao especificada, são gerados
% mes e ano correntes
% \date{maio}{2001}

% o local de realização do trabalho pode ser especificado (ex. para TCs)
% com o comando \location:
\location{Berlin}{Germany}

% itens individuais da nominata podem ser redefinidos com os comandos
% abaixo:
% \renewcommand{\nominataReit}{Prof\textsuperscript{a}.~Wrana Maria Panizzi}
% \renewcommand{\nominataReitname}{Reitora}
% \renewcommand{\nominataPRE}{Prof.~Jos{\'e} Carlos Ferraz Hennemann}
% \renewcommand{\nominataPREname}{Pr{\'o}-Reitor de Ensino}
% \renewcommand{\nominataPRAPG}{Prof\textsuperscript{a}.~Joc{\'e}lia Grazia}
% \renewcommand{\nominataPRAPGname}{Pr{\'o}-Reitora Adjunta de P{\'o}s-Gradua{\c{c}}{\~a}o}
% \renewcommand{\nominataDir}{Prof.~Philippe Olivier Alexandre Navaux}
% \renewcommand{\nominataDirname}{Diretor do Instituto de Inform{\'a}tica}
% \renewcommand{\nominataCoord}{Prof.~Carlos Alberto Heuser}
% \renewcommand{\nominataCoordname}{Coordenador do PPGC}
% \renewcommand{\nominataBibchefe}{Beatriz Regina Bastos Haro}
% \renewcommand{\nominataBibchefename}{Bibliotec{\'a}ria-chefe do Instituto de Inform{\'a}tica}
% \renewcommand{\nominataChefeINA}{Prof.~Jos{\'e} Valdeni de Lima}
% \renewcommand{\nominataChefeINAname}{Chefe do \deptINA}
% \renewcommand{\nominataChefeINT}{Prof.~Leila Ribeiro}
% \renewcommand{\nominataChefeINTname}{Chefe do \deptINT}

% A seguir são apresentados comandos específicos para alguns
% tipos de documentos.

% Relatório de Pesquisa [rp]:
% \rp{123}             % numero do rp
% \financ{CNPq, CAPES} % orgaos financiadores

% Trabalho Individual [ti]:
% \ti{123}     % numero do TI
% \ti[II]{456} % no caso de ser o segundo TI

% Monografias de Especialização [espec]:
% \espec{Redes e Sistemas Distribuídos}      % nome do curso
% \coord[Profa.~Dra.]{Weber}{Taisy da Silva} % coordenador do curso
% \dept{INA}                                 % departamento relacionado

%
% palavras-chave
% iniciar todas com letras minúsculas, exceto no caso de abreviaturas
%
\keyword{formatação eletrônica de documentos}
\keyword{\LaTeX}
\keyword{ABNT}
\keyword{UFRGS}

%\settowidth{\seclen}{1.10~}

%
% inicio do documento
%
\begin{document}

% folha de rosto
% às vezes é necessário redefinir algum comando logo antes de produzir
% a folha de rosto:
% \renewcommand{\coordname}{Coordenadora do Curso}
\maketitle

% dedicatoria
% \clearpage
% \begin{flushright}
%     \mbox{}\vfill
%     {\sffamily\itshape
%       ``If I have seen farther than others,\\
%       it is because I stood on the shoulders of giants.''\\}
%     --- \textsc{Sir~Isaac Newton}
% \end{flushright}

% agradecimentos
%\chapter*{Agradecimentos}
%Agradeço ao \LaTeX\ por não ter vírus de macro\ldots



% resumo na língua do documento
\begin{abstract}
    Este documento é um exemplo de como formatar documentos para o
    Instituto de Informática da UFRGS usando as classes \LaTeX\
    disponibilizadas pelo UTUG\@. Ao mesmo tempo, pode servir de consulta
    para comandos mais genéricos. \emph{O texto do resumo não deve
      conter mais do que 500 palavras.}
\end{abstract}

% resumo na outra língua
% como parametros devem ser passados o titulo e as palavras-chave
% na outra língua, separadas por vírgulas
\begin{englishabstract}{Using \LaTeX\ to Prepare Documents at II/UFRGS}{Electronic document preparation. \LaTeX. ABNT. UFRGS}
    This document is an example on how to prepare documents at II/UFRGS
    using the \LaTeX\ classes provided by the UTUG\@. At the same time, it
    may serve as a guide for general-purpose commands. \emph{The text in
      the abstract should not contain more than 500~words.}
\end{englishabstract}

% lista de figuras
\listoffigures

% lista de tabelas
\listoftables

% lista de abreviaturas e siglas
% o parametro deve ser a abreviatura mais longa
\begin{listofabbrv}{SPMD}
    \item[SMP] Symmetric Multi-Processor
    \item[NUMA] Non-Uniform Memory Access
    \item[SIMD] Single Instruction Multiple Data
    \item[SPMD] Single Program Multiple Data
    \item[ABNT] Associação Brasileira de Normas Técnicas
\end{listofabbrv}

% idem para a lista de símbolos
% \begin{listofsymbols}{$\alpha\beta\pi\omega$}
%     \item[$\sum{\frac{a}{b}}$] Somatório do produtório
%     \item[$\alpha\beta\pi\omega$] Fator de inconstância do resultado
% \end{listofsymbols}

% sumario
\tableofcontents

% aqui comeca o texto propriamente dito

% introducao
\chapter{Introduction}
abc

\section{Details....}

asabasca

\bibliographystyle{abntex2-alf}
\bibliography{biblio}

\end{document}
